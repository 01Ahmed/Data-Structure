\documentclass[11pt]{article}            % Report class in 11 points
\parindent0pt  \parskip10pt             % make block paragraphs
\usepackage{graphicx}
\usepackage{listings}
\graphicspath{ {images/} }
\usepackage{graphicx} %  graphics header file
\begin{document}
\begin{titlepage}
    \centering
  \vfill
    \includegraphics[width=8cm]{uni_logo.png} \\ 
	\vskip2cm
    {\bfseries\Large
	Data Structure and Algorithms \\ 
	
	\vskip2cm
	Lab Report 
	 
	\vskip2cm
	}    

\begin{center}
\begin{tabular}{ l l  } 

Name: & M Ahmed Akram \\ 
Registration \#: & CSU-F16-111 \\ 
Lab Report \#: & 05 \\ 
 Dated:& 30-04-2018\\ 
Submitted To:& Mr. Usman Ahmed\\ 

 %\hline
\end{tabular}
\end{center}
    \vfill
    The University of Lahore, Islamabad Campus\\
Department of Computer Science \& Information Technology
\end{titlepage}


    
    {\bfseries\Large
\centering
	Experiment \# 5 \\

Link list-Basic Deletion at desired position  \\
	
	}    
 \vskip1cm
 \textbf {Objective}\\ 
The objective of this session is to insertion, traversal and deletion at desired position in link list
using C++.\\

\textbf {Software Tool} \\
1.   Dev\ C++


\section{Theory }        
This section discusses how to insert an item into, and delete an item from, a linked list.
Consider the following definition of a node. (For simplicity, we assume that the info type is int.\\
struct nodeType\\
\{\\
int info nodeType* link;\\
\};\\
We will use the following variable nodeType\\
\textbf{INSERTION:- }\\ 
Algorithms which insert nodes into the linked list come up in various situations. We discuss
three of them here. The first one inserts a node at the beginning of the list, the second one inserts
a node after a node with a given location, and the third one inserts a node into the sorted list.\\
\textbf{Inserting at the Beginning of the List:-}
Suppose our linked list is not necessarily sorted and there is no reason to insert a new node
in any special place in the list. Then the easiest place to insert the node is at the beginning of
the list. An algorithm that does so follows.\\

\textbf{Algorithm:- }\\
INSFIRST(INFO,LINK,START,AVAIL,ITEM)\\
This algorithm inserts ITEM as the first node in the list.\\
1.[OVERFLOW?] If AVAIL=NULL then write OVERFLOW and Exit.\\
2.[Remove first node from AVAIL list.] Set NEW=AVAIL
and AVAIL=LINK[AVAIL]\\
3.Set INFO[NEW]=ITEM [Copies new data into new node]\\
4.Set LINK[NEW]=START [New node now points to the original first node]\\
5.Set START=NEW [Change START so it points to the new node ]\\
6.Exit.\\

\textbf{Inserting after a Given Node:- Algorithm:-}\\
INSLOC(INFO,;INK,START,AVAIL,LOC,ITEM)\\
This algorithm inserts ITEM so that item follows the node with location LOC or
inserts ITEM as a first node when LOC=NULL.\\
1.[OVERFLOW?] If AVAIL=NULL then write OVERFLOW and Exit.\\
2.[Remove First Node from the AVAIL list].\\
Set NEW=AVAIL and AVAIL=LINK[AVAIL].\\
3.Set INFO[NEW] =ITEM. [Copies new data into new node.]\\
4.If LOC=NULL then [Inserts as first node] Set\\
LINK[NEW]=START and START=NEW.\\
else [Insert after node with location LOC]\\
Set LINK[NEW]=LINK[LOC] and LINK[LOC]=NEW. [End of
If Structure]\\
5.Exit.\\

\textbf{Inserting into a Sorted Linked List:-}\\
Suppose ITEM is to be inserted into a sorted linked list. Then ITEM must be inserted
between nodes A and B so that\\
\begin{center}
INFO(A) < ITEM < INFO(B)
\end{center}
The following is the procedure which finds the location LOC of node A that is which finds the
location LOC of the last node in the list whose value is less than ITEM. Traverse the list using
pointer variables PTR and comparing ITEM with INFO[PTR] at every node. While traversing
keep track of the location of the preceding node by using a pointer variable SAVE. Thus SAVE
and PTR are updated by assignments.\\

\textbf{Algorithm:-}
FINDA(INFO, LINK START, ITEM,LOC)\\
This procedure finds the location LOC of the last node in a sorted list such that
INFO[LOC] < ITEM or set LOC = NULL.\\
1.[List Empty?] If START = NULL then set LOC= NULL and
Return.\\
2.[Special Case?] If ITEM<INFO[START] then Set LOC =NULL and
Return.\\
3.Set SAVE= START and PTR = LINK[START] [Initialize
Pointers]\\
4.Repeat Step 5 and 6 while PTR ≠ NULL.\\
5. If ITEM<INFO[PTR] then
Set LOC =SAVE and Return. [End of If
Structure]\\
6. Set SAVE =PTR and PTR= LINK[PTR] [Update Pointers]
[End of Step 4 Loop]\\
7.Set LOC=SAVE.\\
8.Exit.\\
Now we have all the components to present an algorithm which inserts ITEM into a
linked list. The simplicity of the algorithm comes from using the previous two
procedures.\\

\textbf{Algorithm:-}
INSERT(INFO, LINK , START, AVAIL, ITEM)\\
This algorithm inserts ITEM into a sorted linked list.\\
1.Call FINDA(INFO, LINK, START, AVAIL , ITEM)\\
2.Call INSLOC(INFO, LINK, START, AVAIL, LOC, ITEM).\\
3.Exit.\\
Consider the linked list shown in Figure 1.\\
\begin{center}
\includegraphics[width=8cm]{1.png}\\
\textbf{Figure : 1}
\end{center}
Suppose that p points to the node with info 65, and a new node with info 50 is to be created
and inserted after p. Consider the following statements:\\
newNode=new nodeType; //create new Node newNode-\\
>info=50; //store 50 in the newnode newNode->link=p-\\
>link;\\
p->link=newNode;\\
\begin{center}
\textbf{Table 1 shows the effect of these statements.\\ Table 1: Inserting a node in a linked list}
\includegraphics[width=8cm]{2.png}
\end{center}

Note that the sequence of statements to insert the node, that is,\\
newNode->link = p->link;\\
p->link = newNode;\\
is very important because to insert newNode in the list we use only one pointer, p, to adjust
the links of the nodes of the linked list. Suppose that we reverse the sequence of the
statements and execute the statements in the following order:\\
p->link = newNode;\\
newNode->link = p->link;\\

\begin{center}
\includegraphics[width=8cm]{3.png}\\
\textbf{Figure : 2}
\end{center}
Figure: List after the execution of the statement p->link = newNode;followed by the
execution of the statement newNode->link = p->link;\\
From Figure, it is clear that newNode points back to itself and the remainder of the list is lost.\\

\section{Lab Task }  
Write a C++ code using functions for the following operations.\\
1.Creating a linked List.\\
2.Traversing a Linked List.\\
3.Inserting the node at the start of the list.\\
4.Inserting a node after a given node.\\
5.Inserting a node in a sorted list.\\
Create a complete menu for the above options and also create option for reusing
it.\\

\subsection{Program:}     

\begin{lstlisting}[language=c++]
#include<iostream>
using namespace std;
struct node
{
    int number;
    node *next;
};

struct Box
{
    void insertAsFirstElement(node *&head, node *&last,int number);
    void insert(node *&head, node *&last,int number);
    void remove(node *&head, node *&last);
    void showList(node *current);
};

bool isEmpty(node *head) {
    return head == NULL;
}

char menu() {
    char choice;
    cout<<"Menu"<<"1. Add an item.\n"<<"2. Remove an item.\n"<<"3. Show the list.\n"<<"Exit\n";
    cin >> choice;
    return choice;
}

void insertAsFirstElement(node *&head, node *&last,int number) {
    node *temp = new node;
    temp->number = number;
    temp->next = NULL;
    head = temp;
    last = temp;
}

void insert(node *&head, node *&last,int number) {
    if(isEmpty(head))
      insertAsFirstElement(head, last, number);
    else {
        node *temp = new node;
        temp->number = number;
        temp->next = NULL;
        last->next = temp;
        last = temp;
    }
}

void remove(node *&head, node *&last,int number) {
    cin>>number;
    node *temp = new node;
    if(isEmpty(head))
        cout<< "this list is already empty.\n";
    else if (head == last) {
        delete head;
        head = NULL;
        last = NULL;
    } else {
        for(int i = 0; 1< number-2; i++)
            /*node *temp = head;
            head = head->next;
            delete temp;*/
            node *temp = head;
        head = head->next;
        temp= temp->next;
        node* temp2 = temp->next;
        temp->next = temp2->next;
        delete temp2;
    }
}

void showList(node *current) {
    if(isEmpty(current))
        cout << "The list is empty\n";
    else {
        cout << "The list contains:\n";
        while(current != NULL) {
            cout << current->number << endl;
            current = current->next;
        }
    }
}

int main() {
    node *head = NULL;
    node *last = NULL;
    char choice;
    int number;
    do{
        choice = menu();
        switch (choice) {
        case'1':
            cout << "please enter a number: ";
            cin >> number;
            insert (head,last,number);
            break;
        case'2':
            remove(head,last,number);
            break;
        case '3':
            showList(head);
            break;
        default:
            cout<< "System exit\n";
        }
    }while (choice !='4');
    return 0;
}

\end{lstlisting}
\begin{center}
 \includegraphics[width=8cm]{4.png}\\
\textbf{Output} 
\end{center}


\section{Conclusion}  
Linked lists provide us the great feature of deleting a node. The process of deletion is also easy to implement. The basic structure is to declare a temporary pointer which points the node to be deleted. Then a little bit of working on links of nodes. 


 
\end{document}                          % The required last line
