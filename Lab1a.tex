\documentclass[11pt]{article}            % Report class in 11 points
\parindent0pt  \parskip10pt             % make block paragraphs
\usepackage{graphicx}
\usepackage{listings}
\graphicspath{ {images/} }
\usepackage{graphicx} %  graphics header file
\begin{document}
\begin{titlepage}
    \centering
  \vfill
    \includegraphics[width=8cm]{uni_logo.png} \\ 
	\vskip2cm
    {\bfseries\Large
	Data Structure and Algorithms \\ 
	
	\vskip2cm
	Lab Report 
	 
	\vskip2cm
	}    

\begin{center}
\begin{tabular}{ l l  } 

Name: & M Ahmed Akram \\ 
Registration \#: & CSU-F16-111 \\ 
Lab Report \#: & 01 \\ 
 Dated:& 16-04-2018\\ 
Submitted To:& Mr. Usman Ahmed\\ 

 %\hline
\end{tabular}
\end{center}
    \vfill
    The University of Lahore, Islamabad Campus\\
Department of Computer Science \& Information Technology
\end{titlepage}


    
    {\bfseries\Large
\centering
	Experiment \# 1 \\

Introduction to Arrays and its operation \\
	
	}    
 \vskip1cm
 \textbf {Objective}\\  The objectives of this lab session are to understand the basic and various operations on arrays in C++. 
 
 \textbf {Software Tool} \\
1.   Dev\ C++


\section{Theory }              
We have already studied array in our computer programming course. We would be using the knowledge we learned there to implement different operation on arrays. \\  \\
\textbf{Traversing Linear Arrays:- }\\ 
Let A be the collection of data elements stored in the memory of the computer. Suppose we want to print the contents of each element of A or suppose we want to count the number of elements of A with a given property. This can be accomplished by traversing A that is by accessing and Processing each element of A exactly once.  \\
The following algorithm traverses a linear array.  The simplicity of the algorithm comes from the fact that LA is a linear structure. Other linear structures such as linked list can also be easily traversed. On the other hand the traversal of non-linear structures such as trees and graphs is considerably more complicated.  \\ \\
\textbf{Algorithm:- }\\
(Traversing a Linear Array) Here LA is a linear Array with lower Bound LB and upper Bound UB.  This algorithm traverses LA.\\
Applying an operation PROCESS to each element of LA. \\
1. Repeat Step 3 and 4 while K<=UB.\\ Visit element Apply PROCESS to LA X.\\ Increase Counter Set X=X+1\\. End of Step 2 Loop\\ 5. Exit. \\ \\

\textbf{Inserting and Deleting: - }\\
Let A be a collection of data elements in the memory of computer. “Inserting” refers to the operation of adding another element to the collection A and “deleting” refers to the operation of removing one of the elements from A. Here we discuss the inserting and deleting when A is a linear array. \\
Inserting an element at the “end” of the linear array can be easily done provided the memory space allocated for the array is large enough to accommodate the additional element. On the other hand suppose we need to insert an element in the middle of the array. Then on average half of the elements must be moved downward to the new location to accommodate the new element and keep the order of other elements.\\
Similarly deleting the element at the “end” of an array presents no difficulties but deleting the element somewhere in the middle of the array would require that each subsequent element be moved one location upward in order to fill up the array.\\ \\

\textbf{Algorithm of Insertion operation: -}\\
1. [Initialize Counter] Set J=N.\\   2. Repeat Step 3 and 4 while J≥K.\\   3. [Move Jth element downward] Set LA [J+1] =LA[J].\\   4. [Decrease Counter] Set J=J-1.   End of Step 2 Loop.\\  5. [Insert element] Set LA[K]=ITEM.\\   6. [Reset N] Set N=N+1. \\  7. Exit.\\ \\ 

\section{Lab Task }  
Write a C++ program to implement all the above described algorithms and display the following menu and ask the user for the desired operation.\\    The program should have the option for reusing it after you have completed the desired task.




\subsection{Program}     

\begin{lstlisting}[language=c++]

#include<iostream.h>
#include<conio.h>
void traversing (int*,int);
void insertion  (int*,int,int,int);
int  deletion   (int*,int,int);
void main()
{ clrscr();
  int array[30],choice,location,item,i,length,position;
  cout<<"Program for Operation of Array by-Tarun Rawat \n";
  cout<<"\nEnter length of array : ";
  cin>>length;
  cout<<"Enter element of array \n";
  for(i=0;i<=length-1;i++)
     { cin>>array[i];
     }
  again:
  cout<<"Travers= 1\nInsert = 2\nDelete = 3\nExit   = 4 \n";
  cout<<"Enter choice: ";
  cin>>choice;
  switch(choice)
        { case 1:       traversing(array,length);
          break;
   case 2:       cout<<"Enter Item and Position to add \n";
          cin>>item>>position;
          insertion(array,length,item,position);
          for(i=0;i<=length;i++)
          {cout<<array[i]<<endl;
          }
          break;
   case 3:       cout<<"Enter position to delete \n";
          cin>>position;
          length =deletion(array,length,position);
          cout<<"New array after delete \n";
          for(i=0;i<length;i++)
      {cout<<array[i]<<"\n";}
          break;
   case 4:       break;
   default:       cout<<"wrong value\n";
    goto again;
        }
getch();
}

void traversing(int array[],int length)
       {  for(int i=0;i<=length-1;i++)
      {cout<<array[i]<<"  ";}
       }


void insertion(int array[],int length,int item,int position)
      { int j=length-1;
        while(j>=position)
      {  array[j+1]=array[j];
        j=j-1;
      }
      array[position]=item;
      length=length+1;
      }

int deletion(int array[],int length,int position)
     {
        while(position<length)
     { array[position]=array[position+1];
       position=position+1;
     }
        length=length-1;
        return length;

      }

\end{lstlisting}

\begin{center}
 \includegraphics[width=8cm]{Capture.png}\\ 
\textbf{Figure : 1 Output}
\vskip 0.5cm
\end{center}

\section{Conclusion}  
Arrays are simple but maybe a bit challenging to understand at first if you don't know how an array is laid out. An array is a contiguous section of data somewhere in RAM (contiguous meaning that it's whole and there are no spaces between the data the array holds). An array is useful for passing large sections of data at time, for instance, in a buffer of audio data or a networking packet. 

 
\end{document}                          % The required last line
