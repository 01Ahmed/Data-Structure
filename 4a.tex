\documentclass[11pt]{article}            % Report class in 11 points
\parindent0pt  \parskip10pt             % make block paragraphs
\usepackage{graphicx}
\usepackage{listings}
\graphicspath{ {images/} }
\usepackage{graphicx} %  graphics header file
\begin{document}
\begin{titlepage}
    \centering
  \vfill
    \includegraphics[width=8cm]{uni_logo.png} \\ 
	\vskip2cm
    {\bfseries\Large
	Data Structure and Algorithms \\ 
	
	\vskip2cm
	Lab Report 
	 
	\vskip2cm
	}    

\begin{center}
\begin{tabular}{ l l  } 

Name: & M Ahmed Akram \\ 
Registration \#: & CSU-F16-111 \\ 
Lab Report \#: & 04 \\ 
 Dated:& 23-04-2018\\ 
Submitted To:& Mr. Usman Ahmed\\ 

 %\hline
\end{tabular}
\end{center}
    \vfill
    The University of Lahore, Islamabad Campus\\
Department of Computer Science \& Information Technology
\end{titlepage}


    
    {\bfseries\Large
\centering
	Experiment \# 4 \\

Link-list Basic Insertion and Traversal   \\
	
	}    
 \vskip1cm

The objective of this session is to understand the various operations on linked list in C++.\\

\textbf {Software Tool} \\
1.   Dev\ C++


\section{Theory }        
\begin{center}
 \includegraphics[width=8cm]{1.png}\\ 
\textbf{Figure : 1}
\vskip 0.5cm
\end{center}
  You have already seen how data is organized and processed sequentially using an array, called
a sequential list. You have performed several operations on sequential lists, such as sorting,
inserting, deleting, and searching. You also found that if data is not sorted, searching for an
item in the list can be very time consuming, especially with large lists. Once the data is sorted,
you can use a binary search and improve the search algorithm. However, in this case, insertion
and deletion become time consuming, especially with large lists because these operations
require data movement. Also, because the array size must be fixed during execution, new items
can be added only if there is room. Thus, there are limitations when you organize data in an
array.\\
This session helps you to overcome some of these problems. Last session showed how memory
(variables) can be dynamically allocated and deallocated using pointers. This session uses
pointers to organize and process data in lists, called linked lists. Recall that when data is stored
in an array, memory for the components of the array is contiguous—that is, the blocks are
allocated one after the other. However, as we will see, the components (called nodes) of a linked
list need not be contiguous.\\ \\
\textbf{LINKED LIST:-  }\\ 
A linked list is a collection of components, called nodes. Every node (except the last node)
contains the address of the next node. Thus, every node in a linked list has two components:
one to store the relevant information (that is, data) and one to store the address, called the link,
of the next node in the list. The address of the first node in the list is stored in a separate location,
called the head or first. Figure 1 is a pictorial representation of a node.  \\ 
\begin{center}
 \includegraphics[width=8cm]{2.png}\\ 
\textbf{Figure : 2}
\vskip 0.5cm
\end{center}
Linked list: A list of items, called nodes, in which the order of the nodes is determined by the
address, called the link, stored in each node.\\
The list in Figure 3 is an example of a linked
list.\\
\begin{center}
 \includegraphics[width=8cm]{3.png}\\ 
\textbf{Figure : 3}
\vskip 0.5cm
\end{center}
The arrow in each node indicates that the address of the node to which it is pointing is stored
in that node. The down arrow in the last node indicates that this link field is NULL.\\
For a better understanding of this notation, suppose that the first node is at memory location
1200, and the second node is at memory location 1575, see Figure
4.\\
\begin{center}
 \includegraphics[width=8cm]{4.png}\\ 
\textbf{Figure : 4}
\vskip 0.5cm
\end{center}
The value of the head is 1200, the data part of the first node is 45, and the link component of
the first node contains 1575, the address of the second node. We will use the arrow
notation whenever we draw the figure of a linked list.\\
For simplicity and for the ease of understanding and clarity, Figures 3 through 5 use decimal
integers as the values of memory addresses. However, in computer memory the memory
addresses are in binary.\\
Because each node of a linked list has two components, we need to declare each node as a
class or struct. The data type of each node depends on the specific application—that is, what
kind of data is being processed. However, the link component of each node is a pointer. The
data type of this pointer variable is the node type itself. For the previous linked list, the definition
of the node is as follows. (Suppose that the data type is int.)

struct nodeType\\
\{\\ 
int info;\\
nodeType* link;\\
\};\\
The variable declaration is as follows:\\
nodeType* head;

\textbf{Linked Lists: Some Properties }\\
To better understand the concept of a linked list and a node, some important properties of
linked lists are described next. \\
Consider the linked list in Figure 5.
\begin{center}
 \includegraphics[width=8cm]{5.png}\\ 
\textbf{Figure : 5}
\vskip 0.5cm
\end{center}

This linked list has four nodes. The address of the first node is stored in the pointer head. Each
node has two components: info, to store the info, and link, to store the address of the next node.
For simplicity, we assume that info is of type int. Suppose that the first node is at location 2000,
the second node is at location 2800, the third node is at location 1500, and the fourth node is at
location 3600. Table 1 shows the values of head and some other nodes in the list shown in Figure
5.\\
\begin{center}
\textbf{Table 1: Values of head and some of the nodes of the linked list in Figure 5}\\
 \includegraphics[width=8cm]{6.png}\\ 
\textbf{Figure : 5}
\vskip 0.5cm
\end{center}

Suppose that current is a pointer of the same type as the pointer head. Then the statement\\
current = head;\\
Copies the value of head into current. Now consider the following
statement:\\
current = current->link;\\
This statement copies the value of current->link, which is 2800, into current. Therefore, after
this statement executes, currentpoints to the second node in the list. (When working with
linked lists, we typically use these types of statements to advance a pointer to the next node in
the list.) See Figure 7.\\
\begin{center}
 \includegraphics[width=8cm]{7.png}\\ 
\textbf{Figure : 7}
\vskip 0.5cm
\end{center}

\begin{center}
\textbf{Table 2 shows the values of current, head, and some other nodes in Figure 7.\\Table 2: Values of current, head, and some of the nodes of the linked list in Figure 7\\}
 \includegraphics[width=8cm]{8.png}\\ 
\textbf{Figure : 8}
\vskip 0.5cm
\end{center}

\textbf{TRAVERSING A LINKED LIST:- }
The basic operations of a linked list are as follows: Search the list to determine whether a
particular item is in the list, insert an item in the list, and delete an item from the list. These
operations require the list to be traversed. That is, given a pointer to the first node of the list,
we must step through the nodes of the list.\\


\textbf{Algorithm:- }\\
(Traversing a Linked List) Let LIST be a linked list in memory. This algorithm traverses
LIST applying an operation PROCESS to each element of LIST. The variable PTR points
to the node currently being processed.\\
1.Set PTR=START [Initialize pointer PTR]\\
2.Repeat Step 3 and 4 while PTR ≠ NULL.\\
3. Apply PROCESS TO INFO[PTR].\\
4. Set PTR=LINK[PTR] [PTR now points to the next node] [End
of Step 2 Loop]\\
5.Exit.\\
Suppose that the pointer head points to the first node in the list, and the link of the last node is
NULL. We cannot use the pointer head to traverse the list because if we use the head to traverse
the list, we would lose the nodes of the list. This problem occurs because the links are in
only one direction. The pointer head contains the address of the first node, the first node
contains the address of the second node, the second node contains the address of the third node,
and so on. If we move head to the second node, the first node is lost (unless we save a pointer
to this node). If we keep advancing head to the next node, we will lose all the nodes of the list
(unless we save a pointer to each node before advancing head, which is impractical because it
would require additional computer time and memory space to maintain the list).\\
Therefore, we always want head to point to the first node. It now follows that we must traverse
the list using another pointer of the same type. Suppose that current is a pointer of the same
type as head. The following code traverses the list:\\
current =head;\\
while(current!=NULL)\\
\{\\
cout<<current->info<<"";\\
current=current->link;\\
\}

\section{Lab Task }  
Write a C++ code using functions for the following operations.\\
1.Creating a linked List.\\
2.Traversing a Linked List.\\
Create a complete menu for the above options and also create option for reusing
it.\\

\subsection{Program:}     
\begin{lstlisting}[language=c++]
#include<iostream>
using namespace std;
struct Node {
    int data;
    Node* next;
};
Node *newNode(int data)
{
    Node *new_node = new Node;
    new_node->data = data;
    new_node->next = NULL;
    return new_node;
}
Node* insertEnd(Node* head, int data)
{
    if (head == NULL) 
         return newNode(data);
    else
        head->next = insertEnd(head->next, data);
    return head;
}
 
void traverse(Node* head)
{
    if (head == NULL)
       return;
    cout << head->data << " ";
 
    traverse(head->next);
}
int main()
{
    Node* head = NULL;
    head = insertEnd(head, 6);
    head = insertEnd(head, 8);
    head = insertEnd(head, 10);
    head = insertEnd(head, 12);
    head = insertEnd(head, 14);
    traverse(head);
    return 0;
}

\end{lstlisting}



\section{Conclusion}  
A linked list is a data structure that can store an indefinite amount of items. These items are connected using pointers in a sequential manner. There are two types of linked list; singly-linked list.


 
\end{document}                          % The required last line
